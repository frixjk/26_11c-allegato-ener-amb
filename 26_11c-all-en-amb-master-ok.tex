%!TEX TS-program = xelatex
%!TEX encoding = UTF-8 Unicode
\documentclass[12pt,twoside,a4paper]{memoir}
\usepackage{fontspec,xltxtra,xunicode}
\defaultfontfeatures{Mapping=tex-text}
\setromanfont[Mapping=tex-text]{TeX Gyre Termes} % va da 12pt
%\setromanfont[Mapping=tex-text]{TeX Gyre Pagella} % basta 11pt
%\setromanfont[Mapping=tex-text]{TeX Gyre Schola} 
%\setromanfont[Mapping=tex-text]{TeX Gyre Bonum} 
\setsansfont[Scale=MatchLowercase,Mapping=tex-text]{TeX Gyre Adventor}
%\setsansfont[Scale=MatchLowercase,Mapping=tex-text]{TeX Gyre Heros}
%\setsansfont[Scale=MatchLowercase,Mapping=tex-text]{Futura}
\setmonofont[Scale=MatchLowercase]{Andale Mono}
%
\usepackage{polyglossia}
\setmainlanguage{italian}
% *********************PACCHETTI******************
\usepackage{lipsum}
\usepackage[colorinlistoftodos,italian,textsize=footnotesize,prependcaption]{todonotes}
\usepackage{paralist}
\usepackage{enumitem}
\usepackage[italian]{varioref}  	%pantieri p. 40 
\usepackage{comment}
\usepackage[output-decimal-marker={,}]{siunitx} % virgola nei decimali
\usepackage{color,soul}
\usepackage{graphicx}

%-----------------------------------
%----------------ULTIMO PACCHETTO---
%-----------------------------------
\usepackage{xcolor}
\definecolor{burgundy}{rgb}{0.5, 0.0, 0.13}
\usepackage{hyperref}% mi serve che ci sia
\hypersetup{%
colorlinks=true,% se = false circonda il testo del link 
linkcolor=burgundy,% comanda anche l'indice
%urlbordercolor=1 0 0 
urlcolor=red,%
citecolor=green, % colora {in verde} i link attivi
pdfauthor={arch. Federico Morchio},%
pdfsubject={regolamento edilizio},%
pdfkeywords={26_11, regolamento edilizio, allegato energetico-ambientale},%
}%
% *************************************
%\setlength{\parindent}{0em} % no indentazione in memoir
% **************************************************
% --------INCLUDE--ONLY--PREAMBLE----------------------
% --------------------------------------------------
\includeonly{% non commentare qua ma nel DOCUMENT
% FRONTMATTER%
0-frontmatter/0-premessa,% no estensione.tex
%<percorso file>,%
%***********************************
% MAINMATTER
%===================================
1-mainmatter/T-1/capitolo,%
1-mainmatter/T-2/capitolo,%
% APPENDIX
%
% BACKMATTER
3-backmatter/colophon,%
}% chiude tutto
%
% *************************************
%*-----------------------------COPERTINA
% *************************************
%----------------------------------
\newcommand*{\plogo}{\fbox{$\mathcal{OA}$}} % logo PROVVISORIO 
% -----------------------------------
% COPERTINA - FRONTESPIZIO - TITLE PAGE --
% -----------------------------------
\newcommand*{\titleGM}{\begingroup % Create the command for including the title page in the document
\hbox{ % Horizontal box
\hspace*{0.2\textwidth} % Whitespace to the left of the title page
\rule{1pt}{\textheight} % Vertical line
\hspace*{0.05\textwidth} % Whitespace between the vertical line and title page text
\parbox[b]{0.75\textwidth}{ % Paragraph box which restricts text to less than the width of the page
{\noindent\huge\bfseries Comune di Casal Cermelli \\[0.5\baselineskip] \Large{Provincia di Alessandria \\ Regione Piemonte}}\\[5\baselineskip] % Title
{\Large\textbf{Regolamento Edilizio Comunale}}\\[2\baselineskip] % Tagline or further description
{\fbox{\Large \textsc{\textcolor{blue}{Allegato Energetico Ambientale }}}} \\[3\baselineskip] % Tagline or further description+ colore by fede
{\Large \textsc{Regolamento Edilizio Comunale}} \\ 

\textit{Progettista:} \\ arch. Federico Morchio % Author name
\medskip\\   \raggedright

{\hspace{13em}\rlap{\raisebox{-1.30\baselineskip}[0pt][0pt]{\includegraphics[angle=0]{timbro-ordine.pdf}}}\\
}
%\textit{Collaboratori:} \\ arch. Andrea Gamondo 
\medskip

{15076 Ovada (AL) -  via Gramsci 109 \\ progettazione@oikosatelier.it - (39) 0143.80233 \par} % Editor affiliation

\vspace{0.08\textheight} % Whitespace between the title block and the publisher
Il resp. del procedimento: geom. Vilmo G. Bovone\\

\vspace{0.12\textheight} % Whitespace between the title block and the publisher

{\noindent oikosatelier \plogo}\date: Ottobre 2016\\[\baselineskip] % Publisher and logo
}}
\endgroup% 
}
%---------------------------
% FINE COPERTINA
% --------------------------
%------------------------------------
% TOC Style
%------------------------------------
\setcounter{tocdepth}{0} % 0 numera PART+Chapter
%\setcounter{secnumdepth}{3}
%\setsecnumdepth{subsection} % va inserito se voglio numerate anche le subsections


%\setsecnumdepth{subsubsection}
\addto\captionsitalian{\renewcommand{\partname}{TITOLO}}  
\addto\captionsitalian{\renewcommand{\chaptername}{Articolo}}  % 
\renewcommand\cftpartname{\partname~} % "part NAME" nel TOC
\renewcommand\cftchaptername{\chaptername~} % "CHAPTER NAME" nel TOC
%
\setlength{\cftpartnumwidth}{3em} % nel TOC allarga la colonna
%
% *********************************************
% -----------------------INIZIO DOCUMENTO
% *********************************************
\begin{document}
%
\frenchspacing % OBBLIGATORIO vedi guida guitt beccari
\tightlists % solo in memoir per restingere spazi nelle liste
%\firmlists 
\pagestyle{empty}  	% COPERTINA --- Removes page numbers
 % per non avere header e footer
\titleGM % copertina tipo mia
% -----------------------------
% STILI DI PAGINA ALTERNATIVI 
% ----------------------------- 
%\chapterstyle{madsen}   	% BEI NUMERONI!
\chapterstyle{lyhne}   		% usalo in caso di titoli lunghi.
%\chapterstyle{ell}  		% occhio ha solo il numero senza Articolo o Capitolo
%\chapterstyle{tandh}
%\makechapterstyle{AlexanderGrebenkov}{%
%  \renewcommand{\chapterheadstart}{\vspace*{\beforechapskip}\hrule\medskip}
%  \renewcommand{\chapnamefont}{\normalfont\large\scshape}
%  \renewcommand{\chapnumfont}{\normalfont\large\scshape}
%  \renewcommand{\chaptitlefont}{\normalfont\large\scshape}
%  \renewcommand{\printchaptername}{Articolo}%{\S}
%  \renewcommand{\chapternamenum}{ }
%  \renewcommand{\printchapternum}{\chapnumfont \thechapter}
%  \renewcommand{\afterchapternum}{. }
%  \renewcommand{\afterchaptertitle}{\par\nobreak\medskip\hrule\vskip
%\afterchapskip}
%}
% -------------------------------------
% ********************************
% -----------------------FRONTMATTER
% ********************************
%
\frontmatter
% !TEX root = ../26_11c-all-en-amb-master-ok.tex

\chapter{Premessa}
\label{chp:premessa}

\section{Principali norme di riferimento}
\label{sec:norme-riferimento-princ}

Il presente documento è redatto ai sensi delle  normative nazionali e piemontesi vigenti al Novembre 2016.

\section{Nota concettuale}
\label{sec:nota-import}

L'allegato Energetico Ambientale ad un Regolamento Edilizio –~a nostro avviso~– deve riportare essenzialmente \emph{i principi di base che regolano la progettazione e la realizzazione} di interventi su edifici – nuovi o esistenti – con attenzione verso gli aspetti della \emph{sostenibilità}, della \emph{bioclimatica}, della \emph{bioarchitettura} e del \emph{benessere complessivo} per i fruitori dei manufatti  – che nella maggioranza dei casi coincidono con i committenti – e delle persone che dall'esterno vivono l'ambiente in cui tali manufatti sono inseriti.

La proposta, alternativa, di un estratto di normative ed un elenco di dati – concepiti come limiti prestazionali e/o descrizione tecnica di sistemi tecnologici – originerebbe importanza marginale e transitoria, poste la variabilità delle disposizioni di legge in materia energetica a cui abbiamo assistito negli ultimi anni e la rapida e costante mutazione prestazionale delle tecnologie connesse al mondo dell'edilizia. È probabile che, in questo modo, l'allegato nascerebbe già vecchio. 

La proposizione di regole – perlopiù fondate sul \emph{buonsenso progettuale} e sul \emph{rispetto di pochi ma fondamentali principi guida} – appare una via più efficace,  fermo restante che questo allegato costituirà una base di riferimento che il Comune potrà implementare  nel tempo senza incorrere in stravolgimenti operativi fastidiosi per i progettisti, per i committenti e per coloro che dovranno verificare le effettive rispondenze delle loro proposte edificatorie al presente regolamento.



 % no estensione .tex
% *************************************
% INSERISCO QUA GLI INDICI
% *************************************
\textcolor{purple}{\listoftodos} %list of todo colorata
\clearpage
\tableofcontents*  % Nb l'asterisco in 'memoir' evita che nell'indice venga scritto anche 'indice' con la relativa pagina.
\clearpage
%\listoftables*    % idem come x TOC
%\newpage
%\listoffigures*    % idem come x TOC
%\clearpage
%
% ********************************
% -----------------------FRONTMATTER
% ********************************
\mainmatter
% !TEX root = ../../26_11c-all-en-amb-master-ok.tex

\chapter{chapter}
\label{chp:----}

\section{sez0}
\label{sec:0}

\lipsum[1-5]


\section{sez-1}
\label{sec:1}

\lipsum[6-12]%
% !TEX root = ../../26_11c-all-en-amb-master-ok.tex

\chapter{Premessa}
\label{chp:premessa}

\section{????}
\label{sec:?}

Il presente documento 


\section{titolo}
\label{sec:nota-import}

\lipsum[20-23]%
% ********************************
% -----------------------APPENDICE
% ********************************
%\appendix
\backmatter
% !TEX root = ../26_11c-all-en-amb-master-ok.tex

% v.4 copyright page - Colophon
\newpage
~\vfill
\thispagestyle{empty}
\setlength{\parindent}{0pt}
\setlength{\parskip}{\baselineskip}

\par \textsc{Comune Di Casal Cermelli (AL)}

Copyright \copyright\ \the\year\ 

\par\textsc{Published by  Comune di Casal Cermelli}
%\vfill

\par Questo Regolamento è stato compilato da Federico Morchio con \XeLaTeX\ utilizzando la classe \emph{memoir}, chapterstyle=\emph{lyhne}, oltre alcune personalizzazioni. \\
Per la composizione finale e la compilazione sono stati utilizzati:

\begin{itemize}[topsep=-3ex,itemsep=-2ex]
\item  il software \emph{TeXShop} (\url{http://pages.uoregon.edu/koch/texshop/}) su \emph{Mac OS X 10.11.6}.\\
\item Font con grazie (tondo): TeX Gyre Termes\\
\item \textsf{Font senza grazie: TeX Gyre Adventor}\\
\item \texttt{Font monospaziato: Andale Mono}
\end{itemize}

I disegni sono stati realizzati da Federico Morchio con: \\ \emph{Inkscape} (\url{https://inkscape.org/en/})

%\vfill 
\par Progetto grafico e impaginazione: Federico Morchio\\
Si ringrazia  il forum del Gruppo Utilizzatori Italiani di \TeX\ per i preziosi consigli (\url{http://www.guitex.org/home/it/forum/index})

\par \href{http://www.comune.casalcermelli.al.it/}{\texttt{www.comune.casalcermelli.al.it/}}

\par \href{http://www.oikosatelier.it}{\texttt{www.oikosatelier.it}}
\vfill

\begin{figure}[h]
\centering
\includegraphics[width=0.3\linewidth]{images/by-nc-sa-eu.jpeg}
%  \checkparity This is an \pageparity\ page.%
%\caption[][6pt]{licenza .... testo.}
\label{fig:textfig-cc-licenza}
%\zsavepos{pos:textfig}
\end{figure}
\par Quest'opera è distribuita con licenza \emph{Creative Commons Attribuzione - Non commerciale 4.0 Internazionale}.
Prima di utilizzarne il contenuto è opportuno comprendere i termini della licenza visionandoli presso il sito:\\ \href{http://creativecommons.org/licenses/by-nc/4.0/}{\texttt{creativecommons.org/licenses/by-nc/4.0/}}
\index{license}
\vfill
\par ISBN (no) \\
%\par \textbf{1}2345678910\\
\texttt{I stampa: Dicembre 2016}
% FINE COLOPHON
%

\end{document}