% !TEX root = ../26_11c-all-en-amb-master-ok.tex

\chapter[Introduzione]{INTRODUZIONE ALL'ALLEGATO ENERGETICO AMBIENTALE}
\label{chp:allegato-en-amb}

\section{Interventi assoggettati all'allegato}
\label{sec:interventi-assogg}

\todo[inline]{vanno valutati bene sulla base delle norme}

I principi contenuti nel presente allegati sono applicabili a tutti i tipi interventi edilizi. Vanno osservati obbligatoriamente per le seguenti tipologie e per quelle ad esse assimilabili in via interpretativa o per specifica deliberazione comunale:

\begin{itemize}
\item nuova costruzione
\item ristrutturazione edilizia
\item sostituzione edilizia
\item ristrutturazione urbanistica
\item ampliamento
\item la sopraelevazione
\item il completamento che presupponga cambio di destinazione o nuovo volume
\item cambio di destinazione d'uso di locali esistenti da trasformare a scopo residenziale o per attività lavorative che presuppongano il soggiorno di persone nei nuovi locali (uffici, negozi e assimilabili)
\end{itemize}

Per gli edifici assoggettati a vincolo ai sensi del codice dei beni architettonici le norme di tutela prevalgono sugli aspetti trattati dal presente allegato.

\section{Contenuti dell'allegato:}
\label{contenutidellallegato:}

L'allegato:

\begin{itemize}
\item riporta indicazioni progettuali di massima. Il progettista le approfondirà in funzione dei propri obiettivi e delle proprie conoscenze

\item definisce alcune linee guida progettuali da seguire per migliorare i livelli prestazionali degli edifici attraverso la progettazione (bioclimatica) più attenta ai principi di sostenibilità;

\item fornisce idee che il progettista potrà sviluppare autonomamente in virtù delle proprie conoscenze, inclinazioni e attitudini;

\item definisce azioni di progettazione specificatamente mirate alla sostenibilità che possono consentire di ottenere incentivi:

\begin{itemize}
\item \emph{economici} che si traducono in risparmi sugli oneri di costruzione

\item \emph{volumetrici} che si traducono in percentuali di incremento rispetto al volume edificabile ottenibile applicando i parametri urbanistici associati alla zona territoriale urbanistica o all'edificio oggetto di intervento

\end{itemize}

\end{itemize}

\section{A cosa serve l'allegato}
\label{acosaservelallegato}

È uno strumento di supporto alla progettazione per tutti gli interventi edilizi: sia quelli assoggettati obbligatoriamente ai suoi contenuti sia gli altri per i quali assume valore di riferimento ed appoggio progettuale.

Per gli interventi elencati in \Sref{sec:interventi-assogg} il progettista ha l'obbligo di seguire le indicazioni progettuali descritte nel presente allegato che vanno considerate alla stessa maniera di quelle contenute nel regolamento edilizio che definiremo classico.

In estrema sintesi l'allegato energetico ambientale fornisce indicazioni e regole che consentono di utilizzare anche gli elementi naturali – radiazione solare, acqua, vento, luce, ombra – come potenziali sistemi di riscaldamento o raffrescamento gratuiti che vanno affiancati ai materiali ed ai sistemi impiantistico adoperati per garantire i rispetto delle norme e dei regolamenti in ambito igienico-sanitario energetico-prestazionale ed edilizio-regolamentare.

In seconda battuta la sua finalità consiste nel definire una serie di azioni progettuali e realizzative che – in virtù delle ricadute sul piano della sostenibilità che ne derivano – consentono al committente di ottenere benefici in termini di qualità abitativa ed eventuali incentivi e sgravi calcolati sulla base di specifiche tabelle commisurate al grado prestazionale aggiuntivo ottenuto a livello progettuale, a livello costruttivo e a livello operativo (nel ciclo di vita).

\section{Come si usa l'allegato}
\label{comesiusalallegato}

Va visto ed interpretato come un \emph{manuale tecnico} che fornisce una sintesi di indicazioni che il progettista ed il committente sceglieranno in relazione agli obiettivi qualitativi che si sono prefissati. 

Concettualmente va considerato come un supporto operativo per il progettista piuttosto che come un ulteriore documento normativo di appesantimento delle procedure per l'ottenimento delle necessarie autorizzazioni ai lavori.

Contiene (anche) l'indicazione di comuni errori progettuali a cui generalmente corrispondono divieti: esattamente come accade per il regolamento classico, pensiamo ad esempio al rispetto delle distanze tra gli edifici o tra pareti finestrate. In questo senso va considerato che a livello progettuale possono essere analizzati e coretti numerosi aspetti che, in caso contrario, non potrebbero essere più affrontati – con gli stessi costi e la stessa efficacia – una volta avviato il cantiere e costruito il volume. La \fref{fig:efficacia-decis-lechner} evidenzia graficamente come, all'interno di un processo edilizio, le decisioni prese nelle prime fasi – assumano  maggiore efficacia in termini controllo e riduzione dei costi di edificazione ed di impatto ambientale. Mentre man mano che la costruzione avanza, fino a percorrere ed esaurire il proprio ciclo di vita il campo delle possibilità e delle alternative si restringe significativamente. \\ Pensiamo, ad esempio, a fattori determinanti come la definizione dimensionale degli spazi e dei volumi, all'orientamento dell'edificio, alla disposizione interna relazionata all'orientamento, e riflettiamo sulle ricadute delle decisioni in ordine al risparmio energetico ed al benessere interno per gli occupanti.

\begin{figure}[htbp]
\centering  % ordine per clip: left, lower right upper
\includegraphics[width=1\textwidth,keepaspectratio]{images/lechner/efficacia-decisioni-fede}
\caption[SHORT]{Rappresentazione dell'efficacia delle decisioni nel processo edilizio.}
\label{fig:efficacia-decis-lechner}
\end{figure}


In questo senso vanno fatte alcune considerazioni semplici ma chiarificatrici. Ipotizziamo che un committente debba valutare due soluzioni progettuali per una stessa una stessa casa in uno stesso luogo:

\begin{itemize}
\item una esposta in modo coerente con il sito e con il percorso solare apparente,

\item una esposta senza prendere in considerazione i criteri del punto precedente.

\end{itemize}

È evidente come per \emph{la stessa casa} il costo della progettazione – generalmente valutato sul prezzo dell'opera – e della realizzazione risulterà sostanzialmente lo stesso. Questo aspetto potrebbe apparire poco significativo: la casa è girata in un modo piuttosto che in un altro, qualcuno potrebbe anche dire che si tratti di una questione di gusti. Ma se consideriamo che la casa esposta meglio fornisce:

\begin{itemize}
\item maggiore apporto termico invernale

\item migliore possibilità di schermatura solare estiva

\item migliori livelli di illuminamento naturale

\item condizioni meno favorevoli per la proliferazione di muffe e condense interne

\item migliori condizioni di vivibilità e quindi di benessere generale

\end{itemize}

diventa più facile, per il committente scegliere tra le due ipotesi progettuali.
