% !TEX root = ../26_11c-all-en-amb-master-ok.tex

\chapter{Premessa}
\label{chp:premessa}

\section{Principali norme di riferimento}
\label{sec:norme-riferimento-princ}

Il presente documento è redatto ai sensi delle  normative nazionali e piemontesi vigenti al Novembre 2016.

\section{Nota concettuale}
\label{sec:nota-import}

L'allegato Energetico Ambientale ad un Regolamento Edilizio –~a nostro avviso~– deve riportare essenzialmente \emph{i principi di base che regolano la progettazione e la realizzazione} di interventi su edifici – nuovi o esistenti – con attenzione verso gli aspetti della \emph{sostenibilità}, della \emph{bioclimatica}, della \emph{bioarchitettura} e del \emph{benessere complessivo} per i fruitori dei manufatti  – che nella maggioranza dei casi coincidono con i committenti – e delle persone che dall'esterno vivono l'ambiente in cui tali manufatti sono inseriti.

La proposta, alternativa, di un estratto di normative ed un elenco di dati – concepiti come limiti prestazionali e/o descrizione tecnica di sistemi tecnologici – originerebbe importanza marginale e transitoria, poste la variabilità delle disposizioni di legge in materia energetica a cui abbiamo assistito negli ultimi anni e la rapida e costante mutazione prestazionale delle tecnologie connesse al mondo dell'edilizia. È probabile che, in questo modo, l'allegato nascerebbe già vecchio. 

La proposizione di regole – perlopiù fondate sul \emph{buonsenso progettuale} e sul \emph{rispetto di pochi ma fondamentali principi guida} – appare una via più efficace,  fermo restante che questo allegato costituirà una base di riferimento che il Comune potrà implementare  nel tempo senza incorrere in stravolgimenti operativi fastidiosi per i progettisti, per i committenti e per coloro che dovranno verificare le effettive rispondenze delle loro proposte edificatorie al presente regolamento.



